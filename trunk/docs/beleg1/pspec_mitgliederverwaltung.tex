% Mitglieder_anzeigen
\paragraph{Mitglieder\_anzeigen}
\begin{tabular}[t]{p{9.5cm}ll}
\textbf{Prozess}: Mitglieder\_anzeigen 	&\textbf{Datum}:      &21.04.2006\\
					&\textbf{Bearbeiter}: &S. Eckelmann\\
\end{tabular}

\hrulefill\\
\textbf{Voraussetzung}: Benutzer muss als Administrator angemeldet sein.
\begin{verbatim}
 BEGIN
   Finde alle in Mitglieder

   IF erfolgreich THEN
      Setze Mitgliederliste := gefundene_Mitglieder
   ELSE
      Setze Fehler 501
 END
\end{verbatim}
\hrulefill

% Mitglied_anlegen
\paragraph{Mitglied\_anlegen}
\begin{tabular}[t]{p{9.5cm}ll}
\textbf{Prozess}: Mitglied\_anlegen  	&\textbf{Datum}:      &21.04.2006\\
					&\textbf{Bearbeiter}: &S. Eckelmann\\
\end{tabular}

\hrulefill\\
\textbf{Voraussetzung}: Benutzer muss als Administrator angemeldet sein. Es darf keinen anderes Mitglied mit gleichen Login geben (von Datenbank abgefangen).
\begin{verbatim}
 BEGIN
   Lege in Mitglieder Mitglied mit Mitgliedsdaten an

   IF gefunden THEN
     Setze Fehler 502
   ELSE
     Setze Bestätigung_Mitgliedanlegen
 END
\end{verbatim}
\hrulefill



% Mitglied_ändern
\paragraph{Mitglied\_ändern}
\begin{tabular}[t]{p{9.5cm}ll}
\textbf{Prozess}: Mitglied\_ändern  	&\textbf{Datum}:      &21.04.2006\\
					&\textbf{Bearbeiter}: &S. Eckelmann\\
\end{tabular}

\hrulefill\\
\textbf{Voraussetzung}: Benutzer ist angemeldet. Mitglied.Mitglieds\_Nr ist die Nummer des aktuell angemeldeten Nutzers (außer er ist Administrator)
\begin{verbatim}
  BEGIN
   Setze in Mitglieder Mitglied_Eintrag
         mit Mitglied_Eintrag.Mitglieds_Nr = Mitglied.Mitglieds_Nr
         auf Mitglied;
  
   IF erfolgreich THEN
     Setze Bestätigung_Mitgliedändern
   ELSE
     Setze Fehler 503
 END
\end{verbatim}
\hrulefill



% Mitglied_löschen
\paragraph{Mitglied\_löschen}
\begin{tabular}[t]{p{9.5cm}ll}
\textbf{Prozess}: Mitglied\_löschen  	&\textbf{Datum}:      &21.04.2006\\
					&\textbf{Bearbeiter}: &S. Eckelmann\\
\end{tabular}

\hrulefill\\
\textbf{Voraussetzung}: Benutzer muss als Administrator angemeldet sein.
\begin{verbatim}
 BEGIN
   Lösche in Mitglieder Mitglied mit Mitglied.Mitglieds_Nr = 
                                     Mitglieds_Nr;

   IF erfolgreich THEN
     Lösche in Kommentare Kommentar mit Kommentar.Mitglieds_Nr = 
                                        Mitglieds_Nr

     Setze Bestätigung_Mitgliedlöschen
   ELSE
     Setze Fehler 504
 END
\end{verbatim}
\hrulefill