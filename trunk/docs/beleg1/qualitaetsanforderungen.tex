\chapter{Spezifikation wichtiger Qualitätsanforderungen}
\section{Benutzerbarkeit}
Im wesentlichen existieren drei Arten von Nutzern: \\
\\
\textbf{1. Administrator:} \\
Er ist im wesentlichen f\"ur die Mitgliedsverwaltung zust\"andig, also das anlegen, \"andern und l\"oschen von Mitgliedseintr\"agen. \\
Des weiteren k\"ummert er sich mit Hilfe der quelloffenen Basisprogramme um Datensicherungen. Eine gewisse Grundkenntnis im Umgang mit
besagten Programmen ist also vorrausgesetzt. \\
\\
\textbf{2. Mitglied:} \\
Nachdem man sich in der Literaturverwaltung registriert hat wird man zum Mitglied. Damit er\"offnen sich neue M\"oglichkeiten. \\
So kann man neue Literatur hinzuf\"ugen, ab\"andern und l\"oschen oder Kommentare \"uber Literatureintr\"age verfassen.\\
\\
\textbf{3. Nutzer:} \\
Wenn man sich nicht registriert ist man in der Gruppe der Nutzer, diese Gruppe hat keine M\"oglichkeit auf den Datenbestand der Literaturverwaltung einzuwirken.\\
Die Rechte sind auf die Recherche beschr\"ankt, der Nutzer kann also B\"ucher suchen, ausw\"ahlen und sich die Beschreibungen und Kommentare ansehen.
\section{Zuverlässigkeit}
Eine einfache Bedienoberfl\"ache und klar formulierte Fehlermeldungen erm\"oglichen dem Nutzer bei Fehleingaben schnell auf die entsprechende Situation zu reagieren.\\
Daten, die gespeichert werden sollen werden erst nach Best\"atigung in den Bestand \"ubernommen, sollte es vor der Best\"atigung zu einem Absturz kommen, muss das Mitglied die Daten erneut eingeben. \\
Sollte es zur unbeabsichtigten L\"oschung oder Besch\"adigung von Daten kommen, hat der Administrator au\ss erhalb des zu entwicklenden Systems M\"oglichkeiten mit Hilfe des Verwendeten Datenbanksystems diese zu retten.
\section{Integrität}
Auch wenn aktiv auf den Datenbestand eingewirkt werden kann, so ist dies jedoch ausschlie\ss lich registrierten Mitgliedern m\"oglich solche \"Anderungen vorzunehmen.
Diese \"Anderungen k\"onnen auch vom Administrator breinigt werden.\\
Sollte es wiederholt zu sch\"adigenden Eingriffen seitens eines Mitglieds kommen, so ist der Adminstrator berechtigt betreffendes Mitgliedskonto samt aller Mitgliedsdaten zu l\"oschen.
\section{Flexibilität}
Auch wenn zur Zeit keine Erweiterungen an der Literaturverwaltung geplant sind, ist es aufgrund der Sprache PHP dennoch m\"oglich \"Anderungen vorzunehmen.
\section{Portabilität}
Aufgrund einer speziellen Konfiguration des Zielsystem in Hinblick auf Web- und SQL-Server, ist das portieren der Literaturverwaltung nicht ohne Weiteres m\"oglich.