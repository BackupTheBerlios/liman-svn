% Kommentar_anlegen
\paragraph{Kommentar\_anlegen}
\begin{tabular}[t]{p{9.5cm}ll}
\textbf{Prozess}: Kommentar\_anlegen  	&\textbf{Datum}:      &21.04.2006\\
					&\textbf{Bearbeiter}: &S. Eckelmann\\
\end{tabular}

\hrulefill\\
\textbf{Voraussetzung}: Benutzer ist angemeldet. Kommentardaten.Mitglieds\_Nr ist die Nummer des aktuell angemeldeten Nutzers. Kommentare ist nicht voll.
\begin{verbatim}
  BEGIN
   Finde in Bibliothek Literatur
         mit Literatur.Titel = Kommentardaten.Buch_Nr
   und Finde in Literatur.Autor = Literaturdaten.Autor;
 
   IF gefunden THEN
   BEGIN
     Finde in Kommentare Kommentar
           mit Kommentar.Mitglieds_Nr = Kommentardaten.Mitglieds_Nr
           und Kommentar.Literatur_Nr = Kommentardaten.Literatur_Nr

     IF gefunden THEN
       Setze Fehler 301
     ELSE
     BEGIN
       Lege in Kommentare Kommentar mit Kommentardaten an
       Setze Bestätigung_Kommentaranlegen
     END

   END
 END
\end{verbatim}
\hrulefill



% Kommentar_ändern
\paragraph{Kommentar\_ändern}
\begin{tabular}[t]{p{9.5cm}ll}
\textbf{Prozess}: Kommentar\_ändern  	&\textbf{Datum}:      &21.04.2006\\
					&\textbf{Bearbeiter}: &S. Eckelmann\\
\end{tabular}

\hrulefill\\
\textbf{Voraussetzung}: Benutzer ist angemeldet. Kommentardaten.Mitglieds\_Nr ist die Nummer des aktuell angemeldeten Nutzers (außer er ist Administrator)
\begin{verbatim}
 BEGIN
   Setze in Kommentare Kommentar_Eintrag
         mit Kommentar_Eintrag.Kommentar_Nr = Kommentar.Kommentar_Nr
         auf Kommentar;
  
   IF erfolgreich THEN
     Setze Bestätigung_Kommentarändern
   ELSE
     Setze Fehler 302
 END
\end{verbatim}
\hrulefill



% Kommentar_löschen
\paragraph{Kommentar\_löschen}
\begin{tabular}[t]{p{9.5cm}ll}
\textbf{Prozess}: Kommentar\_löschen  	&\textbf{Datum}:      &21.04.2006\\
					&\textbf{Bearbeiter}: &S. Eckelmann\\
\end{tabular}

\hrulefill\\
\textbf{Voraussetzung}: Benutzer ist angemeldet. Kommentardaten.Mitglieds\_Nr ist die Nummer des aktuell angemeldeten Nutzers (außer er ist Administrator)
\begin{verbatim}
 BEGIN
   Lösche in Kommentare Kommentar mit Kommenar.Kommentar_Nr = 
                                      Kommentar_Nr;
   
   IF erfolgreich THEN
     Setze Bestätigung_Kommentarlöschen
   ELSE
     Setze Fehler 303
 END
\end{verbatim}
\hrulefill