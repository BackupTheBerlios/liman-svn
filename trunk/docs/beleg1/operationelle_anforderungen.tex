\chapter{Spezifikation der operationellen Anforderungen}
\section{Operationelle Anforderungen an die Daten und Datenbasen}
Für die Abschätzung des Speicherplatzbedarfes für Daten und Speicher wird von den Annahmen ausgegangen,
dass für die Speicherung von in der Regel (bis auf wenige Ausnahmen, für die das gesondert vermerkt ist)\\
         gZahl - 2Byte, rZahl - 4 Byte\\
         Zeichenkette - 1Byte pro Zeichen (der maximalen Länge) erforderlich sind.

TODO:
\begin{tabular}[ht]{|l|l|l|}
\hline
Element & Struktur & Bytes \\
\hline\hline
% Mitglieder
Mitglied  & M\_Name + (Funktion) & $50+14=64$ \\
M\_Name  &  <Vor>Name + <Familien>Name & $25+25=50$ \\
M\_Name  &  ["Vorsitzender" | "Stellvertreter" | "Kassenwart"] & $14$ \\
\hline\hline
Leistung & abc+cde & $19+23=42$ \\
.... & ... & ... \\
\hline
\end{tabular}

\subsection{Speicher (Angaben zum Speicherbedarf in Byte)}
TODO:
\begin{tabular}[ht]{|l|l|l|l|l|}
\hline
Speicher & Struktur & Minimum & Mittel & Maximum \\
\hline\hline
Mitglieder & {Mitglied} & $0*64$ & $50*64=3200$ & $100*64=6400$ \\
Leistungen & {Leistung} & $0*42$ & $50*42=2100$ & $100*42=4200$ \\
\hline
\end{tabular}

\subsection{Weitere Datenflüsse}
TODO:
\begin{tabular}[ht]{|l|l|l|l|l|}
\hline
Datenfluss & Struktur & Minimum & Mittel & Maximum \\
\hline\hline
Revisionsunterlagen & {Mitglied} & $0*64$ & $50*64=3200$ & $100*64=6400$ \\
\hline
\end{tabular}

\subsection{Abschätzungen}
TODO:
\begin{itemize}
 \item \textbf{Gesamtsystem}: blablabla
 \item \textbf{Mitglieder}: blub
 \item \textbf{Leistungen}: jaja
 \item \textbf{Revisionsunterlagen}: nicht die auch noch
\end{itemize}

\subsection{Genauigkeit}
TODO siehe Beispielbeleg S. 23

\section{Operationelle Anforderungen an die Datenströme}
TODO siehe Beispielbeleg S. 23

\section{Operationelle Anforderungen an die Prozesse}
TODO siehe Beispielbeleg S. 23