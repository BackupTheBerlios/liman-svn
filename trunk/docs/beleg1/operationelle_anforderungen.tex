\chapter{Spezifikation der operationellen Anforderungen}
\section{Operationelle Anforderungen an die Daten und Datenbasen}
Für die Abschätzung des Speicherplatzbedarfes für Daten und Speicher wird von den Annahmen ausgegangen,
dass für die Speicherung von in der Regel (bis auf wenige Ausnahmen, für die das gesondert vermerkt ist)\\
         gZahl - 2Byte, rZahl - 4 Byte\\
         Zeichenkette - 1Byte pro Zeichen (der maximalen Länge) erforderlich sind.

\begin{longtable}{|l|p{6.0cm}|p{2cm}|}
\hline
Element & Struktur & Bytes\\
\hline\hline
\endhead

Literatur & @Literatur\_Nr + Art + Titel + Autor + (Jahr) + (Verlag) + (ISBN) + (Beschreibung) + (Ort) + (Stichworte) & $4 + 13 + 40 + 40 + 2 + 40 + 20 + 250 + 40 + 100 = 549$ \\
\hline
Literatur\_Nr & gZahl *6stellig, 000001 $\leq$ Literatur\_Nr $\leq$ 999999*  & $4$\\
\hline
Art & [``Buch'' | ``Artikel'' | ``Handbuch'' | ``Diplomarbeit'' | ``Sonstiges''] \textbf{BITTE UM VORSCHLÄGE/ÄNDERUNG}  & $13$\\
\hline
Titel & Zeichenkette40 & $40$ \\
\hline
Autor & Zeichenkette40 & $40$\\
\hline
Jahr & gZahl *4stellig* & $2$\\
\hline
Verlag & Zeichenkette40 & $40$ \\
\hline
ISBN & Zeichenkette20 & $20$  \\
\hline
Beschreibung & Zeichenkette250 & $250$ \\
\hline
Ort & Zeichenkette40 & $40$ \\
\hline
Stichworte & Zeichenkette100 & $100$\\
\hline\hline

Kommentar & @Kommentar\_Nr + Kommentartext + Literatur\_Nr + Mitglieds\_Nr & $4 + 400 + 4 + 4 = 412 $\\
\hline
Kommentar\_Nr & gZahl *6stellig, 000001 $\leq$ Kommentar\_Nr $\leq$ 999999* & $4$ \\
\hline
Kommentartext & Zeichenkette400 & $400$\\
\hline\hline

Mitglied  & @Mitglieds\_Nr  + Name + Vorname + Login + Passwort + Rechte + E-Mail & $4 + 20 + 20  + 12 + 12 + 13 + 50 = 131 $\\
\hline
Mitglieds\_Nr & gZahl *6stellig, 000001 $\leq$ Benutzer\_Nr $\leq$ 999999* & $4$ \\ 
\hline
Name & Zeichenkette20 & $20$\\
\hline
Vorname & Zeichenkette20 6 $20$ \\
\hline
E-Mail & Zeichenkette30 & $50$ \\
\hline
Login & Zeichenkette12 & $12$\\
\hline
Passwort & Zeichenkette12 & $12$\\
\hline
Rechte & [``Benutzer'' $\mid $``Administrator``] & $13$ \\
\hline\hline

Mitgliedsdaten & Name + Vorname + Login + Passwort + Rechte + E-Mail& $20 + 20 + 12 + 12 + 13 + 50 = 117 $\\
\hline
Literaturdaten & Art + Titel + Autor + (Jahr) + (Verlag) + (ISBN) + (Beschreibung) + (Ort) + (Stichworte) & $13 + 40 + 40 + 2 + 40 + 20 + 250 + 40 + 100 = 545 $ \\
\hline
Kommentardaten & Literatur\_Nr + Mitglieds\_Nr + Kommentartext & $4 + 4 + 4 = 12 $ \\
\hline
BibTeX-Datei & *Datei im Bibtexformat* & $8$\\
\hline
Suchbegriff & Zeichenkette100 & $100 $ \\
\hline
Exportanfrage & Literatur\_Nr \textbf{BITTE UM VORSCHLÄGE/ÄNDERUNG} & $4 $ \\
\hline\hline

Bestätigung\_BibTeXImport & Literatur\_Nr + Zeichenkette ''Importieren vorgenommen`` & $4 + 23 = 27 $\\
\hline
Bestätigung\_Literaturanlegen & Literatur\_Nr + Zeichenkette ''Anlegen vorgenommen`` & $4 + 19 = 23 $\\
\hline
Bestätigung\_Literaturändern & Literatur\_Nr + Zeichenkette ''Änderung vorgenommen`` & $4 + 20 = 24$ \\
\hline
Bestätigung\_Literaturlöschen & Literatur\_Nr + Zeichenkette ''Löschen vorgenommen`` & $4 + 19 = 23 $\\
\hline
Bestätigung\_Kommentaranlegen & Kommentar\_Nr + Zeichenkette ''Anlegen vorgenommen`` & $4 + 19 = 23 $\\
\hline
Bestätigung\_Kommentarändern & Kommentar\_Nr + Zeichenkette ''Änderung vorgenommen`` & $4 + 20 = 23$\\
\hline
Bestätigung\_Kommentarlöschen & Kommentar\_Nr + Zeichenkette ''Löschen vorgenommen`` & $4 + 19 = 23$ \\
\hline
Bestätigung\_Mitgliedanlegen & Mitglieds\_Nr + Zeichenkette ''Anlegen vorgenommen`` & $4 + 19 = 23 $ \\
\hline
Bestätigung\_Mitgliedändern & Mitglieds\_Nr + Zeichenkette ''Änderung vorgenommen`` & $4 + 20 = 24$\\
\hline
Bestätigung\_Mitgliedlöschen & Mitglieds\_Nr + Zeichenkette ''Löschen vorgenommen`` & $4 + 19 = 23$\\
\hline
Literatur\_Info & Literatur + Kommentare & $ 549 + 412 = 961 $ \textbf{KOMMENTARE = VIELE KOMMENTARE UND NICHT NUR EINER}\\
\hline
Suchtreffer & \{Literatur\_Nr + Titel + Autor + Verlag + ISBN\} & $4 + 40 + 40 + 40 + 20 = 144$\\
\hline
Letzte\_Literatur & \{Literatur\_Nr + Titel + Autor + Verlag + ISBN\} & $4 + 40 + 40 + 40 + 20 = 144$\\
\hline
Mitgliederliste & \{Mitglieds\_Nr + Name + Vorname\} & $4 + 20 + 20 = 44 $\\
\hline
\end{longtable}


\subsection{Speicher (Angaben zum Speicherbedarf in Byte)}
TODO:\\
\begin{tabular}[ht]{|l|l|l|l|l|}
\hline
Speicher & Struktur & Minimum & Mittel & Maximum \\
\hline\hline
Literatur & {Literatur} & $0*549$ & $50*549=26950$ & $100*549=54900$ \\
Kommentare & {Kommentar} & $0*412$ & $50*412=20600$ & $100*412=41200$ \\
Mitglieder & {Mitglied}  & $0*381$ & $50*381=19050$ & $100*381=38100$ \\
\hline
\end{tabular}

\subsection{Weitere Datenflüsse}
TODO:\\
\begin{tabular}[ht]{|l|p{3.2cm}|l|l|l|}
\hline
Datenfluss & Struktur & Minimum & Mittel & Maximum \\
\hline\hline

Suchtreffer & \{Literatur\_Nr + Titel + Autor + Verlag + ISBN\}  & $0*144$ & $50*144=7200$ & $100*144=14400$ \\
Letzte\_Literatur & \{Literatur\_Nr + Titel + Autor + Verlag + ISBN\}  & $0*144$ & $50*144=7200$ & $100*144=14400$ \\
BibTeX-Datei & \{Datei im Bibtexformat\} & $0*8$ & $50*8=400$ & $100*8=800$ \\
Revisionsunterlagen &                      &      &            &            \\
 
\\
\hline
\end{tabular}

\subsection{Abschätzungen}
TODO:
\begin{itemize}
 \item \textbf{Gesamtsystem}: Als Nutzungszeit wird von 5 Jahren ausgegangen, wobei die maximale Nutzungszeit, 
die an das System gestellt wird, 10 Jahre betr\"agt.
 \item \textbf{Mitglieder}: Es wird erwartet, dass nicht mehr als 500 Mitglieder den Dienst in Anspruch nehmen werden. Im Durchschnitt wird
 die Mitgliederzahl etwa 200 betragen.
\item \textbf{Leistungen}: Bei einem Stundenaufkommen zwischen 20 und 30 Stunden pro Jahr, kann davon ausgegangen werden, dass jedes 
Mitglied in etwa 50 Suchanfragen, 10 Literatureinstellung und 20 Kommentaren abgibt. Die Informationen bleiben erhalten, solange der Dienst existiert.  
 \item \textbf{Revisionsunterlagen}: Die Revision wird j\"ahrlich vom Administrator durchgef\"uhrt und umfasst normalerweise den Zeitraum seit der letzten Revision.  
\end{itemize}

\subsection{Genauigkeit}
Es werden nur Ganzzahlen im ganzen System verarbeitet. Es werden hierbei keine problematische Rechnungen vorgenommen.

\section{Operationelle Anforderungen an die Datenströme}
Umfangreiche Datenstr\"ome k\"onnen im System nur auftreten, wenn Suchanfragen gestellt, Letzte Literatur aufgerufen, eine BibTex-Datei abgefragt 
oder die Revisionsunterlagen abgerufen werden. Diese Datenstr\"ome bewegen sich im Bereich von maximal ..... Megabyte und im Mittel ...... Kilobyte, sodass es selten
zu gr\"o\"seren Datenstr\"omen kommt.


\section{Operationelle Anforderungen an die Prozesse}
Es wird von Antwortzeien bis zu 5 Sekunden gerechnet. Da die meisten Prozesse auf dem Server verarbeitet werden und der Client diese nur Darstellen muss, 
kann von 5 Sekunden ausgegangen werden. 
