\chapter{Spezifikation der operationellen Anforderungen}
\section{Operationelle Anforderungen an die Daten und Datenbasen}
Für die Abschätzung des Speicherplatzbedarfes für Daten und Speicher wird von den Annahmen ausgegangen,
dass für die Speicherung von in der Regel (bis auf wenige Ausnahmen, für die das gesondert vermerkt ist)\\
         gZahl - 2Byte, rZahl - 4 Byte\\
         Zeichenkette - 1Byte pro Zeichen (der maximalen Länge) erforderlich sind.

\begin{longtable}{|l|p{6.0cm}|p{2cm}|}
\hline
Element & Struktur & Bytes\\
\hline\hline
%\endhead

Literatur & @Literatur\_Nr + Art + Titel + Autor + (Jahr) + (Verlag) + (ISBN) + (Beschreibung) + (Ort) + (Stichworte) & $4 + 13 + 40 + 40 + 2 + 40 + 20 + 250 + 40 + 100 = 549$ \\
\hline
Literatur\_Nr & gZahl *6stellig, 000001 $\leq$ Literatur\_Nr $\leq$ 999999*  & $4$\\
\hline
Art & [``Buch'' | ``Artikel'' | ``Handbuch'' | ``Diplomarbeit'' | ``Sonstiges''] \textbf{BITTE UM VORSCHLÄGE/ÄNDERUNG}  & $13$\\
\hline
Titel & Zeichenkette40 & $40$ \\
\hline
Autor & Zeichenkette40 & $40$\\
\hline
Jahr & gZahl *4stellig* & $2$\\
\hline
Verlag & Zeichenkette40 & $40$ \\
\hline
ISBN & Zeichenkette20 & $20$  \\
\hline
Beschreibung & Zeichenkette250 & $250$ \\
\hline
Ort & Zeichenkette40 & $40$ \\
\hline
Stichworte & Zeichenkette100 & $100$\\
\hline\hline

Kommentar & @Kommentar\_Nr + Kommentartext + Literatur\_Nr + Mitglieds\_Nr & $4 + 400 + 4 + 4 = 412 $\\
\hline
Kommentar\_Nr & gZahl *6stellig, 000001 $\leq$ Kommentar\_Nr $\leq$ 999999* & $4$ \\
\hline
Kommentartext & Zeichenkette400 & $400$\\
\hline\hline

Mitglied  & @Mitglieds\_Nr  + Name + Login + Passwort + Rechte + E-Mail & $4 + 20 + 12 + 12 + 13 = 381 $\\
\hline
Mitglieds\_Nr & gZahl *6stellig, 000001 $\leq$ Benutzer\_Nr $\leq$ 999999* & $4$ \\ 
\hline
Name & <Vor>Name + <Nach>Name & $10 + 10 = 20$\\
\hline
Login & Zeichenkette12 & $12$\\
\hline
Passwort & Zeichenkette12 & $12$\\
\hline
Rechte & [``Benutzer'' $\mid $``Administrator``] & $13$ \\
\hline\hline

Mitgliedsdaten & Name + Login + Passwort + Rechte + E-Mail& $20 + 12 + 12 + 13 + 320= 377 $\\
\hline
Literaturdaten & Art + Titel + Autor + (Jahr) + (Verlag) + (ISBN) + (Beschreibung) + (Ort) + (Stichworte) & $13 + 40 + 40 + 2 + 40 + 20 + 250 + 40 + 100 = 545 $ \\
\hline
Kommentardaten & Literatur\_Nr + Mitglieds\_Nr + Kommentartext & $4 + 4 + 4 = 12 $ \\
\hline
BibTeX-Datei & *Datei im Bibtexformat* & $8$\\
\hline
Suchbegriff & [Zeichenkette100 | (Art) + (Titel) + (Autor) + (ISBN) + (Jahr) + (Beschreibung) + (Ort) + (Stichworte)] *Mindestens ein optionales Feld muss angegeben sein* & $100 $ \\
\hline
Exportanfrage & Literatur\_Nr \textbf{BITTE UM VORSCHLÄGE/ÄNDERUNG} & $4 $ \\
\hline\hline

Bestätigung\_BibTeXImport & Literatur\_Nr + Zeichenkette ''Importieren vorgenommen`` & $4 + 23 = 27 $\\
\hline
Bestätigung\_Literaturanlegen & Literatur\_Nr + Zeichenkette ''Anlegen vorgenommen`` & $4 + 19 = 23 $\\
\hline
Bestätigung\_Literaturändern & Literatur\_Nr + Zeichenkette ''Änderung vorgenommen`` & $4 + 20 = 24$ \\
\hline
Bestätigung\_Literaturlöschen & Literatur\_Nr + Zeichenkette ''Löschen vorgenommen`` & $4 + 19 = 23 $\\
\hline
Bestätigung\_Kommentaranlegen & Kommentar\_Nr + Zeichenkette ''Anlegen vorgenommen`` & $4 + 19 = 23 $\\
\hline
Bestätigung\_Kommentarändern & Kommentar\_Nr + Zeichenkette ''Änderung vorgenommen`` & $4 + 20 = 23$\\
\hline
Bestätigung\_Kommentarlöschen & Kommentar\_Nr + Zeichenkette ''Löschen vorgenommen`` & $4 + 19 = 23$ \\
\hline
Bestätigung\_Mitgliedanlegen & Mitglieds\_Nr + Zeichenkette ''Anlegen vorgenommen`` & $4 + 19 = 23 $ \\
\hline
Bestätigung\_Mitgliedändern & Mitglieds\_Nr + Zeichenkette ''Änderung vorgenommen`` & $4 + 20 = 24$\\
\hline
Bestätigung\_Mitgliedlöschen & Mitglieds\_Nr + Zeichenkette ''Löschen vorgenommen`` & $4 + 19 = 23$\\
\hline
Literatur\_Info & Literatur + Kommentare & $ 549 + 412 = 961 $ \textbf{KOMMENTARE = VIELE KOMMENTARE UND NICHT NUR EINER}\\
\hline
Suchtreffer & \{Literatur\_Nr + Titel + Art + Beschreibung\} & $4 + 40 + 13 + 250 = 307$\\
\hline
Mitgliederliste & \{Mitglieds\_Nr + Name\} & $4 + 20 = 24 $\\
\hline
\end{longtable}


\subsection{Speicher (Angaben zum Speicherbedarf in Byte)}
TODO:\\
\begin{tabular}[ht]{|l|l|l|l|l|}
\hline
Speicher & Struktur & Minimum & Mittel & Maximum \\
\hline\hline
Literatur & {Literatur} & $0*549$ & $50*549=26950$ & $100*549=54900$ \\
Kommentare & {Kommentar} & $0*412$ & $50*412=20600$ & $100*412=41200$ \\
Mitglieder & {Mitglied}  & $0*381$ & $50*381=19050$ & $100*381=38100$ \\
\hline
\end{tabular}

\subsection{Weitere Datenflüsse}
TODO:\\
\begin{tabular}[ht]{|l|l|l|l|l|}
\hline
Datenfluss & Struktur & Minimum & Mittel & Maximum \\
\hline\hline

Suchtreffer & \{Literatur\_Nr + Titel + Art + Beschreibung\}  & $0*307$ & $50*307=15350$ & $100*307=30700$ \\
BibTeX-Datei & \{Datei im Bibtexformat\} & $0*8$ & $50*8=400$ & $100*8=800$ \\
 
\\
\hline
\end{tabular}

\subsection{Abschätzungen}
TODO:
\begin{itemize}
 \item \textbf{Gesamtsystem}: blablabla
 \item \textbf{Mitglieder}: blub
 \item \textbf{Leistungen}: jaja
 \item \textbf{Revisionsunterlagen}: nicht die auch noch
\end{itemize}

\subsection{Genauigkeit}
Es werden nur Ganzzahlen im ganzen System verarbeitet. Es werden hierbei keine problematische Rechnungen vorgenommen.

\section{Operationelle Anforderungen an die Datenströme}
TODO siehe Beispielbeleg S. 23

\section{Operationelle Anforderungen an die Prozesse}
TODO siehe Beispielbeleg S. 23