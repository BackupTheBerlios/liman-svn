\chapter{Basismaschine und Entwicklungsumgebung}
\section{Basismaschine}
\subsection{Hardwarekonfiguration}
F"ur {\bf Webserver} wird folgende Ausstattung ben"otigt:
\begin{itemize}
 \item Server (Monitor wird nicht ben"otigt)
 \item Breitband-Netzwerk (z. B. SDSL, Ethernet)
 \item 5 MB freien Festplattenspeicher
\end{itemize}
Als notwendige Minimalausstattung des für den {\bf Nutzer-PC} wird vorgeschlagen:
\begin{itemize}
 \item Basis-PC (Tastatur, Maus, Monitor)
 \item Netzwerk (z. B. Modem, Ethernet)
\end{itemize}
Da die Nutzung des Systems komplett Serverseitig funktioniert wird auf dem Nutzer-PC kein Festplattenspeicher verbraucht, da die
Bedienung komplett "uber den Webbrowser erfolgt. Eine schnelle CPU und eine breitbandige Netzwerkverbindung
kann den Arbeitskomfort zus"atzlich verbessern.


\subsection{Betriebssystem}
Webserver:
\begin{itemize}
\item Unix oder Unix-kompatibel
\end{itemize}
Nutzer-PC:
\begin{itemize}
\item Windows ab Windows 95
\item Linux oder Unix-kompatibel
\item MacOS
\item PalmOS
\end{itemize}

\subsection{Sonstige Basissoftware}
Webserver:
\begin{itemize}
\item MySQL 5.0
\item Apache 2.0
\item PHP 5.1
\end{itemize}
Nutzer-PC:
\begin{itemize}
\item Webbrowser, z. B. Firefox, Konqueror, Opera, Internet Explorer
\end{itemize}


\section{Entwicklungsumgebung}
Es wird ein normaler Texteditor verwendet (z. B. vim, emacs).
