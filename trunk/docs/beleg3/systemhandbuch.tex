\chapter{Systemhandbuch}
\section{Installationsanleitung}
Das System ist vollkommen frei und kostenlos "uber das Internet von \pdfurl{http://liman.berlios.de/} 
bzw. direkt der 
Downloadseite \pdfurl{http://developer.berlios.de/project/showfiles.php?group\_id=6441} 
zu beziehen.

Je nach verf"ugbarem Entpackprogramm sollte man die neuste Version im .tar.gz, 
.tar.bz2 (zu entpacken mit z.b. GNU tar, \pdfurl{http://www.gnu.org/software/tar/}) oder .zip-Format 
(zu entpacken mit z. B. WinZIP, \pdfurl{http://www.winzip.de/}) herunterladen und entpacken.

Nun muss der Inhalt des Archivs auf das Server in einen vom Webserver verwalteten Ordner 
abgelegt werden, z. B. in (oder unterhalb) von /var/www auf einem Standard-Debian System.

Befindet man sich nicht direkt auf dem Webserver, k"onnen die Dateien beispielsweise mit 
einem FTP-Programm oder einem SCP-Programm hochgeladen werden, je nach dem was der Server 
unterst"utzt.

Die Datenbankverbindung und der erste Administrator werden "uber das Installationskript 
festgelegt. Dazu ist es empfohlen, dass der Webserver Schreibzugriff auf include/config.php
hat. Es sollte also ggf. noch schreibbar gesetzt werden. Richten sie nun Ihren Webbrowser 
(je nach Installationsort) auf http://ihr.server.de/liman/install.php und tragen sie die 
Verbindungsdaten zum MySQL-Server ein.
Prefix beschreibt die ersten Zeichen der Tabellennamen und sollte im Allgemeinen so belassen
werden.

Mit dem Best"atigen des Formulares werden automatisch die ben"otigten Tabellen in der Datenbank 
und der Administrator angelegt. Sie k"onnen ihre Einstellungen auch sp"ater noch in der Datei
include/config.php anpassen.

"Uberpr"ufen sie Ihre Einstellungen mit einem Login auf http://ihr.server.de/liman/ .
Wenn alles korrekt installiert wurde, sollten abschlie�end noch die Datei install.php 
aus dem Webserver-Verzeichnis gel"oscht werden, damit niemand die Datenbankverbindungsdaten neu 
setzen kann.

\section{Programm-Filesystem}
{TODO}
\begin{longtable}{|l|p{6.0cm}|p{2cm}|}
\hline
{\bf Bezeichnung der Datei} & {\bf Bedeutung} & {\bf Anzahl Bytes}\\
\hline
\endhead

index.php & Startseite mit Letzte Literatur & 1196\\
\hline
login.php & login/logout mit Begr"u"sungstext & 836\\
\hline
user.php & Nutzerdetails & 1680\\
\hline
usermode.php & Nutzer l"oschen/"andern/hinzuf"ugen & 7892\\
\hline
userlist.php & Nutzerdaten anzeigen & 1924\\
\hline
search.php & Suche (Ergebnisse) & 1893\\
\hline
searchmore.php & Suche (Formular) & 1284\\
\hline
litmod.php & Literatur l"oschen / "andern / hinzuf"ugen & 10056\\
\hline
commentmod.php & Kommentar l"oschen / "andern / hinzuf"ugen & 3089\\
\hline
lit.php & Literaturdetails & 4436\\
\hline
bibtex.php & BibTeX Importieren & 2318\\
\hline
include$\backslash$config.php & Datenbankkonfiguration & 362\\
\hline
include$\backslash$global.php & Einrichtung Prog. - Umgebung & 1557\\
\hline
include$\backslash$literatur.php & Stellt Funktionen f"ur Literatur in Bibliothek zur Verf"ugung & 11846\\
\hline
include$\backslash$mitglied.php & Stellt Funktionen f"ur Mitglieder bereit & 8436\\
\hline
include$\backslash$suche.php & Durchsucht die Tabelle Bibliothek und Autoren "uber Volltextsuche & 7852\\
\hline
include$\backslash$footer.php & Ende jeder Html - Datei & 36 \\
\hline
include$\backslash$header.php & Kopf jeder Html-Datei & 2885 \\
\hline
include$\backslash$literaturart.php & Speicherung aus der Datenbank oder BibTeX-Datei erhaltenen Art von Informationen & 3811\\
\hline
include$\backslash$sqldb.php & Datenbankzugriffsauswahl & 354 \\
\hline
include$\backslash$autor.php & Stellt Funktionen f"ur Autoren bereit & 4608\\
\hline
include$\backslash$formhelper.php & Hilfsfunktionen f"ur Formulare & 2278\\
\hline
include$\backslash$kommentar.php & Stellt Funktionen f"ur Kommentare bereit & 7911\\
\hline   
include$\backslash$login.php &  Verwaltet die Session des Users und der damit entstehenden Rechte & 5243\\
\hline
include$\backslash$sqldbmysql.php & Startet eine Verbindung zum Datenbanksystem her und w"ahlt eine Datenbank & 6323\\
\hline

\end{longtable}


\section{Administrationsanleitung}
