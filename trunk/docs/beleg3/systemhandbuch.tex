\chapter{Systemhandbuch}
\section{Installationsanleitung}
Das System ist frei und kostenlos "uber das Internet von \pdfurl{http://liman.berlios.de/} 
bzw. direkt der 
Downloadseite \pdfurl{http://developer.berlios.de/project/showfiles.php?group\_id=6441} 
zu beziehen.

Je nach verf"ugbarem Entpackprogramm sollte man die neuste Version im .tar.gz, 
.tar.bz2 (zu entpacken mit z.b. GNU tar, \pdfurl{http://www.gnu.org/software/tar/}) oder .zip-Format 
(zu entpacken mit z. B. 7-Zip, \pdfurl{http://www.7-zip.org/}) herunterladen und entpacken.

Nun muss der Inhalt des Archivs auf das Server in einen vom Webserver verwalteten Ordner 
abgelegt werden, z. B. in (oder unterhalb) von /var/www auf einem Standard-Debian System.

Befindet man sich nicht direkt auf dem Webserver, k"onnen die Dateien beispielsweise mit 
einem FTP-Programm oder einem SCP-Programm hochgeladen werden, je nach dem was der Server 
unterst"utzt.

Die Datenbankverbindung und der erste Administrator werden "uber das Installationskript 
festgelegt. Dazu ist es empfohlen, dass der Webserver Schreibzugriff auf include/config.php
hat. Es sollte also ggf. noch schreibbar gesetzt werden. Richten sie nun Ihren Webbrowser 
(je nach Installationsort) auf http://ihr.server.de/liman/install.php und tragen sie die 
Verbindungsdaten zum MySQL-Server ein.
Prefix beschreibt die ersten Zeichen der Tabellennamen und sollte im Allgemeinen so belassen
werden.

Mit dem Best"atigen des Formulares werden automatisch die ben"otigten Tabellen in der Datenbank 
und der Administrator angelegt. Sie k"onnen ihre Einstellungen auch sp"ater noch in der Datei
include/config.php anpassen.

"Uberpr"ufen sie Ihre Einstellungen mit einem Login auf http://ihr.server.de/liman/ .
Wenn alles korrekt installiert wurde, sollten abschlie"send noch die Datei install.php 
aus dem Webserver-Verzeichnis gel"oscht werden, damit niemand die Datenbankverbindungsdaten neu 
setzen kann.

\section{Programm-Filesystem}



Nachdem sie das System erfolgreich eingerichtet haben, sind folgende Datein auf Ihrem Webverzeichnis:

\begin{longtable}{|l|p{6cm}|r|}
\hline
{\bf Bezeichnung der Datei} & {\bf Bedeutung} & {\bf Anzahl Bytes}\\
\hline\hline
\endhead

index.php & Startseite mit Letzte Literatur & 1364\\
\hline
login.php & login/logout mit Begr"u"sungstext & 1384\\
\hline
user.php & Nutzerdetails & 1983\\
\hline
usermod.php & Nutzer l"oschen/"andern/hinzuf"ugen & 9534\\
\hline
userlist.php & Nutzerdaten anzeigen & 2251\\
\hline
search.php & Suche (Ergebnisse) & 2245\\
\hline
searchmore.php & Suche (Formular) & 1257\\
\hline
litmod.php & Literatur l"oschen / "andern / hinzuf"ugen & 11701\\
\hline
commentmod.php & Kommentar l"oschen / "andern / hinzuf"ugen & 4009\\
\hline
lit.php & Literaturdetails & 4771\\
\hline
bibtex.php & BibTeX Importieren & 2856\\
\hline
design/liman.css & Stylesheet für LiMan & 2097\\
\hline
include/config.php & Datenbankkonfiguration & 362\\
\hline
include/global.php & Einrichtung Prog. - Umgebung & 1375\\
\hline
include/literatur.php & Stellt Funktionen f"ur Literatur in Bibliothek zur Verf"ugung & 13372\\
\hline
include/mitglied.php & Stellt Funktionen f"ur Mitglieder bereit & 9370\\
\hline
include/suche.php & Durchsucht die Tabelle Bibliothek und Autoren "uber Volltextsuche & 8463\\
\hline
include/footer.php & Ende jeder HTML-Datei & 36 \\
\hline
include/header.php & Kopf jeder HTML-Datei & 3236 \\
\hline
include/literaturart.php & Speicherung aus der Datenbank oder BibTeX-Datei erhaltenen Art von Informationen & 3959\\
\hline
include/sqldb.php & Datenbankzugriffsauswahl & 481 \\
\hline
include/autor.php & Stellt Funktionen f"ur Autoren bereit & 5495\\
\hline
include/formhelper.php & Hilfsfunktionen f"ur Formulare & 2278\\
\hline
include/kommentar.php & Stellt Funktionen f"ur Kommentare bereit & 9342\\
\hline   
include/login.php &  Verwaltet die Session des Users und der damit entstehenden Rechte & 5414\\
\hline
include/sqldb\_mysql.php & Startet eine Verbindung zum Datenbanksystem her und w"ahlt eine Datenbank & 7212\\
\hline
\end{longtable}


\section{Administrationsanleitung}


Es wird empfohlen, regelm"a"sig eine Sicherung der Datenbank anzulegen. Das System bietet diese Funktion nicht direkt an,
sondern verweist auf das Programm phpMyAdmin (\pdfurl{http://www.phpmyadmin.net/}). Zust"andig f"ur die Sicherung der Datenbank ist der Link ``Exportieren'' in der Datenbankansicht.


Ihr Datenbank besteht aus mehreren Tabellen, bevor diese Exportiert werden k"onnen, m"ussen sie ausw"ahlen, welche der Tabellen Sie Sichern wollen. 
Daneben stellen Sie ein, ob nur die Struktur 
(also der Aufbau) der jeweiligen Tabelle, die Struktur zusammen mit den gespeicherten Datens"atzen oder nur die Datens"atze gesichert werden sollen.

Weiter unten k"onnen Sie die Funktion Drop Table ausw"ahlen, welche bewirkt, dass die jeweilige Tabelle nach dem Kopieren sofort gel"oscht wird. Des weiteren 
kann man einstellen, ob vollst"andige oder erweiterte Inserts ausgef"uhrt werden sollen.  

Wenn Sie Senden ausw"ahlen, wird die Ausgabe als Datei gespeichert, was f"ur den Zweck der Datensicherung am Sinn vollsten ist. Ansonsten wird die 
Ausgabe auf dem Bildschirm wiedergegeben. 
Falls sie "ofters Sicherungen anlegen, ist eine weitere Option ZIP / GZIP Komprimieren sehr praktisch. Wenn Sie dies ausw"ahlen wird die Datei mit ZIP komprimiert.

Eine weitere Möglichkeit, besonders für die automatisierte Sicherung der Datenbank, bietet das Programm mysqldump, welches ein Teil der MySQL-Anwendungen ist. Eine genau Erklärung über die Funktionsweise von mysqldump ist in den Handbüchern zu MySQL enthalten (\pdfurl{http://www.mysql.com/doc/de/mysqldump.html}).


Eine Fehlermeldung k"onnte zum Beispiel ''Could not connect to SQL server. Please inform Webmaster'' sein. Hier gelingt es dem System nicht, sich mit 
der Datenbank zu verbinden. Als erstes sollten Sie sicher stellen, ob Ihr MySQL-Server gestartet ist. Ein weiterer Grund f"ur diese Fehlermeldung k"onnte 
eine unaktuelle include/config.php sein. In dieser Datei werden alle Einstellungen gespeichert, unter anderem auch f"ur den Datenbankzugriff wichtige Daten. Es 
kann vorkommen, dass hier zum Beispiel ein veralteter Benutzername oder Passwort enthalten ist.

Was sollte man tun, wenn die Fehlermeldung, dass Daten nicht geschrieben/geändert oder gelöscht werden k"onnen, auftritt? Schauen Sie nach, ob die Rechte 
der einzelnen Tabellen der Datenbank richtig gesetzt sind. Am besten mit dem weiter vorn schon erw"ahnten Programm ''phpMyAdmin''.

Des weiteren kann auftreten, dass ein User nach dem Einloggen direkt wieder aus geloggt wird. Der Grund f"ur diesen Fehler ist, dass Ihr Webserver keine 
Schreibrechte auf den Ordner hat, in der PHP die Sessions speichert.