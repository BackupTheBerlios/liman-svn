\chapter{Systemhandbuch}
\section{Installationsanleitung}
Das System ist vollkommen frei und kostenlos "uber das Internet von \pdfurl{http://liman.berlios.de/} 
bzw. direkt der 
Downloadseite \pdfurl{http://developer.berlios.de/project/showfiles.php?group\_id=6441} 
zu beziehen.

Je nach verf"ugbarem Entpackprogramm sollte man die neuste Version im .tar.gz, 
.tar.bz2 (zu entpacken mit z.b. GNU tar, \pdfurl{http://www.gnu.org/software/tar/}) oder .zip-Format 
(zu entpacken mit z. B. WinZIP, \pdfurl{http://www.winzip.de/}) herunterladen und entpacken.

Nun muss der Inhalt des Archivs auf das Server in einen vom Webserver verwalteten Ordner 
abgelegt werden, z. B. in (oder unterhalb) von /var/www auf einem Standard-Debian System.

Befindet man sich nicht direkt auf dem Webserver, k"onnen die Dateien beispielsweise mit 
einem FTP-Programm oder einem SCP-Programm hochgeladen werden, je nach dem was der Server 
unterst"utzt.

Die Datenbankverbindung und der erste Administrator werden "uber das Installationskript 
festgelegt. Dazu ist es empfohlen, dass der Webserver Schreibzugriff auf include/config.php
hat. Es sollte also ggf. noch schreibbar gesetzt werden. Richten sie nun Ihren Webbrowser 
(je nach Installationsort) auf http://ihr.server.de/liman/install.php und tragen sie die 
Verbindungsdaten zum MySQL-Server ein.
Prefix beschreibt die ersten Zeichen der Tabellennamen und sollte im Allgemeinen so belassen
werden.

Mit dem Best"atigen des Formulares werden automatisch die ben"otigten Tabellen in der Datenbank 
und der Administrator angelegt. Sie k"onnen ihre Einstellungen auch sp"ater noch in der Datei
include/config.php anpassen.

"Uberpr"ufen sie Ihre Einstellungen mit einem Login auf http://ihr.server.de/liman/ .
Wenn alles korrekt installiert wurde, sollten Sie abschliessend noch die Datei install.php 
aus dem Webserver-Verzeichnis gel"oscht werden, damit niemand die Datenbankverbindungsdaten neu 
setzen kann.

\section{Programm-Filesystem}

\section{Administrationsanleitung}
