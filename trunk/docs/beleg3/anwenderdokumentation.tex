\chapter{Anwenderdokumentation}
\section{Produktzweck}
Eines der wohl wichtigsten Werkzeugen nicht nur im Studium, sondern 
auch in Berufen speziell in der Forschung und Entwicklung, sind B"ucher. 
Allerdings ist der Umfang von Fachliteratur nicht unbedingt klein und 
"uberschaubar. Au"serdem kann ein Fehlkauf sehr schnell ins Geld gehen, 
bei B"ucherpreisen von 50 EUR und mehr.\\
Wie w"are es also mit einem Werkzeug um Informationen "uber diese 
Werkzeuge "ubersichtlich bereit zu stellen.\\
Informationen wie Titel, Autor und ISBN werden von 
Migliedern verfasst und in einer Datenbank abgelegt. Aber 
damit nicht genug, es ist jedem Mitglied auch m"oglich Kommentare zu 
verfassen, z.B. wie ihm das Buch gefallen hat oder ob es bei Recherchen "uber gewisse Themen
hilfreich war oder nicht.\\
Um einen universellen Zugriff auf die Datenbest"ande der Literaturverwaltung zu gew"ahrleisten, 
erfolgt der Zugriff "uber dynamisch erzeugte HTML-Seiten, die einfach 
mit jedem Web-Browser "uber die Web-Adresse aufgerufen werden k"onnen.\\
Ein weiterer Vorteil, den dynamisch erzeugte HTML-Seiten bieten ist die freie Plattformwahl durch den Nutzer.
\section{Basismaschine und Ressourcenanforderungen}
Die Literaturverwaltung wird von einem Administrator auf einem Web-Server installiert und konfiguriert, dies erm"oglicht jedem Nutzer einen ortsunab"angigen Zugriff auf die Datenbest"ande.\\
Es wird lediglich ein Web-Browser wie z.B. Firefox, Opera oder der Internet Explorer ben"otigt.\\
Die Anforderungen an die Hardware richten sich nach den Anforderungen des benutzten Web-Browsers.\\
\section{Nutzerklassen}
Es gibt drei Nutzerklassen:
\begin{enumerate}
\item Nutzer:\\
Nutzer ist jeder, der sich "uber einen Web-Browser auf die Seite begibt um Informationen abzurufen.\\
Der Nutzer besitzt keinerlei M"oglichkeiten Datenbest"ande anzulegen, zu ver"andern oder zu l"oschen. Er besitzt lediglich Lesezugriff auf die Buchinformationen und Kommentare. \\

\item Mitglied:\\
Das Mitglied ist ein registrierte Nutzer, der sich durch sein Anmelden "uber die grafische Oberfl"ache mit einem LogIn-Namen und einem Passwort als solcher authentifiziert.\\
Als Mitglied darf man zus"atzlich zu den Rechten eines Nutzers neue Buchtitel anlegen oder bereits bestehende Titel bearbeiten und l"oschen.\\
Desweiteren besitzt jedes Mitglied das Recht, Kommentare zu verfassen sowie eigens erstellte Kommentare zu bearbeiten und zu l"oschen.\\
Was die Mitgliedsdaten betrifft, so darf jedes Mitglied die eigenen Nutzerdaten bearbeiten. Was die Rechte betrifft, so ist eine "Anderung ausschlie"slich durch einen Administrator m"oglich.\\

\item Administrator:\\
Zun"achst wird durch einen Administrator die Literaturverwaltung auf einem Web-Server installiert und konfiguriert.\\
Weiterhin ist der Administrator f"ur die Datenverwaltung und inbesondere f"ur die Mitgliedsverwaltung zust"andig, also z.B. f"ur die Pflege der Mitgliedsdaten.\\
Was die Rechte eines Administrators betrifft so sind diese allein auf Grund seines Aufgabengebietes unbeschr"ankt, d.h. er kann alle Daten sehen, erstellen, bearbeiten und l"oschen.\\ 

\end{enumerate}
\section{Bedienungsanleitung}
